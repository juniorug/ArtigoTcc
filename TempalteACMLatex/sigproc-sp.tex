% THIS IS SIGPROC-SP.TEX - VERSION 3.1
% WORKS WITH V3.2SP OF ACM_PROC_ARTICLE-SP.CLS
% APRIL 2009
%
% It is an example file showing how to use the 'acm_proc_article-sp.cls' V3.2SP
% LaTeX2e document class file for Conference Proceedings submissions.
% ----------------------------------------------------------------------------------------------------------------
% This .tex file (and associated .cls V3.2SP) *DOES NOT* produce:
%       1) The Permission Statement
%       2) The Conference (location) Info information
%       3) The Copyright Line with ACM data
%       4) Page numbering
% ---------------------------------------------------------------------------------------------------------------
% It is an example which *does* use the .bib file (from which the .bbl file
% is produced).
% REMEMBER HOWEVER: After having produced the .bbl file,
% and prior to final submission,
% you need to 'insert'  your .bbl file into your source .tex file so as to provide
% ONE 'self-contained' source file.
%
% Questions regarding SIGS should be sent to
% Adrienne Griscti ---> griscti@acm.org
%
% Questions/suggestions regarding the guidelines, .tex and .cls files, etc. to
% Gerald Murray ---> murray@hq.acm.org
%
% For tracking purposes - this is V3.1SP - APRIL 2009

\documentclass{acm_proc_article-sp}
%\usepackage{[utf8x][inputenc]}
\usepackage[english]{babel}
\usepackage[utf8]{inputenc}
\usepackage{enumitem}
\usepackage{minted}
\usepackage{listings}
\usepackage{graphicx,url}
\newcommand{\subparagraph}{}
\usepackage{titlesec}
\setcounter{secnumdepth}{4}


\titleformat{\paragraph}
{\normalfont\normalsize\bfseries}{\theparagraph}{1em}{}
\titlespacing*{\paragraph}
{0pt}{3.25ex plus 1ex minus .2ex}{1.5ex plus .2ex}
\begin{document}

\title{An Extendable Python Library To Manipulate Sensors Coupled To The Raspberry Pi}

\numberofauthors{2} 
\author{
% 1st. author
\alignauthor
Edivaldo M. F. de Jesus Jr\titlenote{Aluno do curso de An\'alise e Desenvolvimento de Sistemas(ADS)}\\
       \affaddr{Instituto Federal da Bahia}\\
       \affaddr{Rua Em\'idio dos Santos, S/N}\\
       \affaddr{Barbalho, Salvador Bahia}\\
       \email{juniorug@gmail.com}
% 2nd. author
\alignauthor
Manoel C. M. Neto\titlenote{Doutor em Ci\^encia da Computa\c{c}\~ao e Professor do Curso de An\'alise e Desenvolvimento de Sistemas}\\
       \affaddr{Instituto Federal da Bahia}\\
       \affaddr{Rua Em\'idio dos Santos, S/N}\\
       \affaddr{Barbalho, Salvador Bahia}\\
       \email{manoelnetom@ifba.edu.br}
}

\date{13 April 2015}

\maketitle
\begin{abstract}

This article encourages the use of open source technologies, due to the growth of free movement of hardware, inferring the possibility of an alternative for engineers and professionals to develop their projects and provide the dissemination of knowledge.
The convergence of radio technologies, microprocessors and personal digital electronic devices is leading to the concept of ubiquitous computing in which intelligent, mobile and stationary devices, coordinate with each other to provide for users immediate and universal access to new services transparently, aimed at increasing human capabilities. This work is intended to be inserted in this context and aims to define, implement and validate the design and implementation of an extensible Python library for manipulating sensors / actuators coupled to the Raspberry Pi using the raspberry-GPIO-python module. The library uses the Abstract Factory pattern to ensure that sensors / actuators and events from the same family being used in conjunction with guaranteed way. On other platforms, such as Arduino, the APIs provide libraries that encapsulate the complexity of implementation and offer only the interface to use. These libraries do not yet exist formally for those who want to use Pyton as development language applied to the Raspberry Pi.  The project also presents the results obtained using some of the implemented sensors, system modeling and results described and analyzed.

\end{abstract}
\keywords{Ubiquitous Computing,Internet of Things, Raspberry Pi, Sensor, Python} % NOT required for Proceedings

\section{Introduction}
The UbiComp in its various ramifications and applications, is considered by many as the twenty-first century's new paradigm of computing. It is the area of computing that studies the coupling of the physical world to the world of information and provides an abundance of services and applications, allowing users, machines, data and objects of physical space interact with each other seamlessly. The theme is considered one of the great challenges of research in Computer Science by the National Science Foundation (NSF) [1] and is also present in Computing search Grand Challenges report in Brazil from 2006 to 2016 [2], published by the Brazilian Computer Society (SBC).
\newline
\newline
Researches on ubiquitous computing are being held on topics such as: basic access to any wireless device, mobility support within the network transparently, safety, context treatment, efficient use of energy, presentation of multimedia content and so on. This work is focused on building intelligent interactive environments. In these environments, the fundamental idea is to create ways to avoid that the user needs to go to the computer/device, allowing many of these working at a distance. The use of platforms to integrate the devices that make up these environments is one of the key points to create it. Currently there are some options to fill this gap. This text emphasizes one: the project Raspberry Pi [3,4].
\newline
\newline
Raspberry Pi is a widely used platform for professionals who are interested in the field of ubiquitous applications. It provides basic interfaces for creating small projects and / or for those who must be fed by battery. The platform allows you to use high-level programming languages that are quite widespread as C / C ++, Python and Java. The Raspberry allows the development of a range of projects. For example, home automation (turn on and off electrical devices, remote control for TV, Air-conditioners, etc.), eBook and Audiobook readers and so on.
\newline
\newline
If we seek a way to access the Raspberry Pi bus we shall find a lot of papers in Python. Which incidentally, brings the acronym of the device name (Pi from Python). However, we cannot find any library that abstracts the details of wiring a sensor or actuator to the Raspberry allowing the user takes care just in read the data already processed and/or converted.
\newline
\newline
The developer must have a degree of experience that can be considered basic to a computer professional, but advanced for those who are not.
\newline This paper is structured as follow:
\begin{itemize}
\item Section 2 present several technologies involved in the preparation of this monograph, aimed to introduce important concepts of Computer area in which the context of the project is inserted
\item Section 3 shows correlated works
\item Section 4 presents the details of the executed implementation
\item Sections 5 and 6 address the testing methodologies used and the results obtained during system validation. 
\item The last section presents the conclusions.
\end{itemize}

\section{THEORETICAL BACKGROUND}

\subsection{General Context}
The term Ubiquitous Computing was first defined by Mark Weiser [33] in the late 80. At this time, Weiser predicted an increasing in functionality and availability of computing services to end users and, on the other hand, he predicted a decreased visibility of these services. For Weiser, the computing would not be exclusive of a computer. He believed that in the future there would be several different devices connected to each other. At a time when users were using PCs (desktops) and that knowledge needed to operate a computer,  Weiser bet on a future where the focus of the users would be the task itself, and not the tool used. In this way, they would use the computer without realizing or require specific technical knowledge. [34]
\begin{figure}[h]
    \includegraphics[width=0.4\textwidth,natwidth=610,natheight=642]{pictures/era.png}
    \caption{Ages of Computing}
\end{figure}
\newline
\newline
The passage of time has shown that betting Weiser was right. For Weiser, [35] evolution of computing has gone through two ages to reach to ubiquitous computing. The first was called the age of mainframe, where many people shared the same computer. The second era was the PC, where each computer was used by one person. Currently, the evolution of distributed information systems, the network connections type options'  expansion, mobile computing and the various types of applications on non-conventional computing devices, are just some of the examples that can confirm: the Ubiquitous Computing (the third age) is already a reality. Figure 1 shows a chart with the ages of computing.
\newline
\newline


Terms such as ubiquitous computing, pervasive computing, nomadic computing, invisible computing, mobile computing and many others, have been used often interchangeably, although they differ conceptually and employing different organization of ideas and management of computer services. Insofar as each area progresses, these concepts will be better understood and its definitions will become clearer. This section presents the key concepts needed to understand the UbiComp besides presenting some project examples in the literature.

\subsection{Mobile Computing}
Mobile computing is based on the ability of a user to load or move (physically) computer services wherever it moves. In this context, the computer becomes an ever-present device that expands the ability of a user to use the services it offers, regardless of their location. Combined with the ability to access the network, mobile computing has transformed the computing an activity that can be taken almost anywhere.
\newline
\newline
An important conceptual limitation of mobile computing is that the computational model used in most applications does not change while users are moving. It means that a device is not able to obtain information on the physical context in which computation occurs, and consequently also can not adapt to the new context correctly. A solution to accommodate the changing context would pass to users the responsibility to monitor and manually configure an application / device to the extent that it moves. However, this solution is not well accepted by most users. This limitation was one of the inspirations for pervasive computing .

\subsection{Pervasive Computing}
The concept of pervasive computing implies that the computer is embedded invisibly in the environment to the user [24]. In this conception, the computer has the ability to: i) obtain environmental information in which it is embedded and ii) use it to build dynamically computational models that allow you to control, configure and tune the application to better suit the needs of a device or user. For this to be possible, the key point is the ability of computers be able to act as "smart" in the environment where users move. This environment is usually populated by computational sensors and services.

\subsection{Ubiquitous Computing}
As can be seen in Figure 2, the UbiComp can be defined as a computer area positioned between the Mobile Computing and Pervasive Computing [14, 24]. Ubiquitous is an adjective originated from Latin (ubiquu) which means "that is at the same time everywhere". Ubiquitous computing benefits from the advances in mobile and pervasive computing and arises from the need to integrate mobility with the functionality of pervasive computing. The term ubiquitous computing will be used here as a junction of pervasive computing and mobile computing. The justification to perform a distinction of these terms is a device that is embedded in an environment, not necessarily is mobile.
\begin{figure}[h]
    \includegraphics[width=0.4\textwidth,natwidth=610,natheight=642]{pictures/ubiq.png}
    \caption{Ubiquitous Computing: intersection between pervasive and mobile computing}
\end{figure}
\newline
\newline
Research in Ubiquitous Computing approach about the technologies and infrastructure that enable the deployment of ubiquitous applications through a number of issues including the following:

\begin{itemize}
\item how to design hardware and operating systems for sensor platforms?
\item how to allow devices to find each other and to use your services?
\item how to allow systems involving limited processing resources and energy, to work well?
\end{itemize}

Generally, ubiquitous applications receive sensor data from other service providers devices, manage user actions, provide support mobility and use context information to perform tasks [6]. A ubiquitous system itself has a set of requirements, peculiarities and challenges that influence the design, implementation, deployment and evaluation of its project. [24] These are cornerstones of UbiComp and quite different from those used in the development of systems for PC’s. Among these points, we can cite as an example.:

\begin{enumerate}
\item Resource-Constrained Devices; 	
\item Volatile Execution Environments; 	
\item Heterogeneous Execution Environments;
\item Fluctuating Usage Environments;
\item Invisible Computing;
\item Security and Privacy;
\end{enumerate}


\subsection{Internet of Things}
The Internet of Things (IoT) is a multidisciplinary field, covering a wide range of subjects, from purely technical issues (eg, routing protocols, semantic queries) to a mixture of technical and social problems (security, privacy, usability) as well as social and business topics. The existing internet of things’ applications are potentially diverse. Monitoring of environmental and personal health, monitoring and control of industrial processes, including agriculture, smart spaces and smart cities are just some examples of the IoT applications [5].
\newline
\newline
The connection of physical things to the Internet makes it possible to access remote sensor data and to control the physical world from a distance. The mash-up of captured data with data retrieved from other sources, e.g., with data that is contained in the Web, gives rise to new synergistic services that go beyond the services that can be provided by an isolated embedded system. The Internet of Things is based on this vision. A smart object, which is the building block of the Internet of Things, is just another name for an embedded system that is connected to the Internet[H. Kopetz, Real-Time Systems. Design Principles for Distributed Embedded
Applications. Kluwer Academic Publications, 1997].
\newline
\newline
Everyday physical things that are enhanced by a small electronic device to provide local intelligence and connectivity to the cyberspace established by the internet. The small electronic device, a \textit{computational component} that is attached to a \textit{physical thing}, bridges the gap between the physical world and the information world. A \textit{smart object} is thus a \textit{cyber-physical system} or an \textit{embedded system} consisting of a \textit{thing} (the physical entity) and a \textit{component} (the computer) that processes the sensor data and supports a wireless communication link to the Internet[H. Kopetz, Real-Time Systems. Design Principles for Distributed Embedded
Applications. Kluwer Academic Publications, 1997].

\subsection{Hardware}
One of the fundamental requirements for developing ubiquitous systems is the use of hardware such as sensors, microcontrollers, communication devices (network cards, Bluetooth, etc.) and storage, among others. For example, sensors allow transform use of interactive environments from more transparent interfaces. Currently, there are some platforms that allow insert and control various types of sensors, using of communication interfaces and storage units. This section presents and details the main hardware devices available for the development of ubiquitous systems.


\subsubsection{Sensors and Actuators}
Sensors are devices that allow us to capture information from the environment in which they are inserted, such as temperature, pressure, presence, humidity, smoke detector, light intensity, among others. In general, the sensors work transforming parts of a physical quantity into an electrical signal, which in turn can be interpreted by electronic devices [7]. In other words, sensors are components that allow an electronic device to interact with the real world.
\newline
\newline
According to [7], when the sensors operate directly, transforming one form of energy into another are called transducers. The sensors where operations occur indirectly alter their properties, such as resistance, capacitance or inductance, under the action of physical grandeur so that this change is roughly proportional. For example, the light sensor LDR (Light-dependent resistors) vary inversely its resistance the amount of light falling on it. Thus, when there is a large amount of light falling on the sensor, they have a very low resistance and this allows the flow of electric current increases, whereas when there is little light, they have a high resistance and prevent current flow.
\newline
\newline
An actuator as well as a sensor is a transducer that converts one form of energy into another, and can also do the opposite [7]. In other words, rather than just  transform parts of a physical quantity into an electrical signal, it can transform an electrical signal into a physical quantity such as motion, magnetism, heat, among others. For example, the relays are electromechanical devices that work with small power, but are able to control external circuits that involve high currents. They are basically composed of a coil and a set of contacts. When a current flows through the coil it creates a magnetic field that attracts and closes the contacts, remaining as long as power supply in the coil. As a result, it allows the passage of energy through the relay.

\subsubsection{Arduino}
Arduino was created in 2005 by Massimo Banzi and David Mellis in Italy with the goal of use as an electronic learning tool and programming for design students, so that they would use in art projects, interactivity and robotics. Electronic learning was expensive: a microcontroller was costing 100 euros. So they decided to make their own board. Sought employees and thus created an efficient technology, accessible and compatible with Windows, Mac and Linux.
\newline
\newline
Arduino is a platform that popularizes the concept of free hardware. In the book Getting Started with Arduino, Massimo Banzi describes the Arduino as a physical open-source computing platform based on a simple board with input/output pins that implements the Processing language. It is a small but powerful board constituted by a microcontroller that can be easily programmed via a Universal Serial Bus interface (USB) and able to build electronic devices and interesting systems.
\newline
\newline
Working with the construction of a hardware requires much time and effort, because is necessary to create new circuits, use various components such as resistors and capacitors and many welds. Using Arduino board abstracts much of this construction process, making it simpler, allowing people from different fields of knowledge being able to build their projects. The platform is widely used worldwide for offering advantages such as:


\begin{itemize}
\item A multiplatform environment that can run all major operating systems such as Windows, Linux and MacOS.
\item Is an open-source hardware, ie, the circuit design is available so that if someone is interested in creating your own card, just buy the necessary components.		
\item Hardware cost is low.					
\item Is possible program it via a USB cable instead of a serial port. Remember that today's computers do not have serial ports.	
\item It has a development environment with intuitive interface for easy use.
\end{itemize}

The fact that both the hardware and the Arduino software be developed in an open, patent-free, allows its projects to be recreated in different ways. The adoption of open hardware concept motivates those who create projects to contribute with functions and libraries for the Arduino. Is knowledge about knowledge, the same principle of free software.
\newline
\newline
Note that there is not necessary advanced knowledge in electronics to use the platform. But for those who are interested in deepening the knowledge in this area, there are several available materials that can help, for example, in [19].

\subsubsection{BeagleBone}
The BeagleBoard is a low-power open-source hardware single-board computer produced by Texas Instruments in association with Digi-Key and Newark element14. The BeagleBoard was also designed with open source software development in mind, and as a way of demonstrating the Texas Instrument's OMAP3530 system-on-a-chip.[7] The board was developed by a small team of engineers as an educational board that could be used in colleges around the world to teach open source hardware and software capabilities. It is also sold to the public under the Creative Commons share-alike license. The board was designed using Cadence OrCAD for schematics and Cadence Allegro for PCB manufacturing; no simulation software was used.
\newline\newline
BeagleBone is an \$89 Manufacturer suggested retail price (MSRP), credit-card-sized Linux computer that connects to the Internet and runs software such as Android 4.0 and Ubuntu. With plenty of I/O and processing power for real-time analysis provided by an AM335x 720MHz ARM\textregistered\, processor, BeagleBone can be complemented with cape plug-in boards to augment functionality.

\subsubsection{Raspberry Pi}
Raspberry Pi is a small computer, approximately with the size of a credit card. It can be used to do many things that are done by a common personal computer (RASPBERY PI FUNDATION, 2014a). In addition, it can also be used in electronic projects because it has a hardware interface: The general purpose input/output port (GPIO). The Raspberry Pi Foundation, creator of the project, is an educational charity headquartered in the UK and aims to help and encourage the teaching of computer science in schools.
\newline
\newline
This computer came up with the intention of reconnecting children and youth in computer programming and stimulate the creation of new projects so there is not only the consumption of the technology created by the market. While in the 1990s there was a growth in the number of children and youth who developed programming skills, starting in the 2000s, this number started to decline. (Raspbery PI Foundation, 2014b).
\newline
\newline
With the realization that, year by year, the students were moving away from programming and reducing the development of skills in computer science. The researchers Eben Upton, Rob Mullins, Jack Lang and Alan Mycrof, from the Computer Laboratory of University of Cambridge had the idea of creating a platform that would allow to reconcile the programming students  and handling computers. (DENNIS, 2013) From 2006 to 2008, early versions of what is now the Raspberry Pi was developed. From 2008, with the expansion and emphasis on mobile devices, the devices are becoming more efficient and cheaper which enabled the launch of the Raspberry Pi model B to the public in February 2012 for \$ 35.
\newline
\newline
The proposal of the Raspberry Pi is therefore be a low-cost computer, with the ability to interact with the outside world through the sensor coupling. As seen previously, it was thought to be used in the educational environment, in order to assist and encourage the teaching of programming and help understand the operation of computers. However, the Raspberry Pi has been used by people of all ages and interests in various projects, for example, projects involving automation, sensing and robotics, games, multimedia, networks and servers.
\newline
\newline
Currently there are three  Raspberry Pi models: The model A, which costs \$ 25 and the B and B+ models, which costs \$ 35. As for the price, there is not much difference between the hardware models. The main differences are the amount of USB ports (1, 2 and 4 in the models A, B and B +, respectively), the Ethernet port (model A does not have) and ram (256MB on the model A against 512 for the others).
%\begin{figure}[h]
%    \includegraphics[width=0.5\textwidth,natwidth=610,natheight=642]{pictures/RaspiModelB.png}
%    \caption{Raspberry Pi Components, model B. Font: (RASPBERY PI FUNDATION, 2014c)
%}
%\end{figure}

\paragraph{Features}
The original Raspberry Pi is based on the Broadcom BCM2835 system on a chip (SoC),[1] which includes an ARM1176JZF-S 700 MHz processor, VideoCore IV GPU,[8] and was originally shipped with 256 megabytes of RAM, later upgraded (models B and B+) to 512 MB.[9] The system has Secure Digital (SD) (models A and B) or MicroSD (models A+ and B+) sockets for boot media and persistent storage.[10]
\newline
\newline
In 2014, the Raspberry Pi Foundation launched the Compute Module, which packages a BCM2835 with 512 MB RAM and an eMMC flash chip into a module for use as a part of embedded systems.[11]
The Foundation provides Debian and Arch Linux ARM distributions for download.[12] Tools are available for Python as the main programming language, with support for BBC BASIC (via the RISC OS image or the Brandy Basic clone for Linux),[14] C, C++, Java, Perl and Ruby.
\newline
\newline
Some other features presents on RPI:
\begin{itemize}
\item Display Serial Interface Connector (DSI): This connector receives a flat-ribbon cable of 15 pins that can be used to communicate with an LCD or organic light-emitting diode (OLED) display screen;	
\item Camera Serial Interface(CSI) Connector: This port allows a camera module be directly coupled to the card;
\item P2 and P3 connectors: These two rows of connectors are JTAG connectors for test to Broadcom chip (P2) and LAN9512 network (P3). Due to the proprietary nature of Broadcom chipset, these connectors are unlikely to be of much used;
\item Pin Protection: Most of the pins in the header go directly to the Broadcom chip. It is important to carefully design the components you attach to them as there is a risk you will permanently damage your Pi. Short circuits and wiring mistakes could also ruin your day so double check everything. A multimeter is probably going to help a lot here as you can double check wiring before you connect to the Pi.
\end{itemize}

 
\paragraph{Power Supply}
The device is powered by 5v micro USB. Exactly how much current (mA) the Raspberry Pi requires is dependent on what you hook up to it. We have found that purchasing a 1.2A (1200mA) power supply from a reputable retailer will provide you with ample power to run your Raspberry Pi for most applications, though you may want to get a 2.5A (2500mA) if you want to use all 4 USB ports on the Model B without using an external powered USB hub.
\newline
\newline
The power requirements of the Raspberry Pi increase as you make use of the various interfaces on the Raspberry Pi. The GPIO pins can draw 50mA safely (that is 50mA distributed across all the pins! An individual GPIO pin can only safely draw 16mA), the HDMI port uses 50mA, the camera module requires 250mA, and keyboards and mice can take as little as 100mA or over 1000mA! Check the power rating of the devices you plan to connect to the Pi and purchase a power supply accordingly. If you’re not sure, buy a powered hub.

\paragraph{Processor}
The System on a chip(SoC) used in the first generation Raspberry Pi is somewhat equivalent to the chip used in older smartphones (such as iPhone / 3G / 3GS). The Raspberry Pi is based on the Broadcom BCM2835 system on a chip (SoC),[1] which includes an 700 MHz ARM1176JZF-S processor, VideoCore IV GPU,[8] and RAM. It has a Level 2 cache of 128 KB, used primarily by the GPU. The SoC is stacked underneath the RAM chip, so only its edge is visible.
%[1]"BCM2835 Media Processor; Broadcom". Broadcom.com. 1 September 2011. Retrieved 6 May 2012.http://www.broadcom.com/products/BCM2835
%[8]Brose, Moses (30 January 2012). "Broadcom BCM2835 SoC has the most powerful mobile GPU in the world?". Grand MAX. Archived from the original on 13 April 2012. Retrieved 13 April 2012.https://web.archive.org/web/20120413184701/http://www.grandmax.net/2012/01/broadcom-bcm2835-soc-has-powerful.html
%[9] "Model B now ships with 512MB of RAM". Raspberrypi.org. Retrieved 15 October 2012.http://www.raspberrypi.org/model-b-now-ships-with-512mb-of-ram/
%[10] "Verified USB Peripherals and SDHC Cards;". Elinux.org. Retrieved 6 May 2012.http://elinux.org/RPi_VerifiedPeripherals
%[11] http://www.raspberrypi.org/raspberry-pi-compute-module-new-product/
%[14] "Brandy Basic". Jaguar.orpheusweb.co.uk. 26 July 2005. Retrieved 6 May 2012. http://jaguar.orpheusweb.co.uk/branpage.html

\paragraph{General Purpose Input/Output}
Another important aspect of hardware from Raspberry is the group of GPIO pins. These pins are programmable ports to input and output data used to provide an interface between the board and peripherals, microcontrollers/microprocessors, sensors, actuators, etc. The GPIO interface is fundamental for building intelligent interactive environments. It is the interface between the Raspberry and the real world. In simple terms, you can consider the GPIO pins as switches that can be turned on/off.
\newline
\newline
Similarly to the Arduino, besides the GPIO, the Raspberry also supports PWM(Pulse Width Modulation), UART(Universal asynchronous receiver/transmitter) and SPI(Serial Peripheral Interface). Of the 26 pins available, 17 are reserved for GPIO and 8 are used for power and ground. Figure 2.6 shows the GPIO interface. The pins can be used via code written in a programming compatible language like Python, Scratch, Java, among others [8].
The GPIO pins are available on the PCB via a header and allow you to interface the Pi to the real world.

\paragraph{Inputs and Outputs}
The number and options of input and output from a Raspberry Pi depend on the model used. Models RPi A+, B+ and 2B GPIO J8 have 40-pin as pinout. Models A and B have only the first 26 pins.In this paper we will present only the pins availlable in the model B.The pins are divided into:
%http://www.raspberrypi.org/documentation/usage/gpio/README.md

\begin{itemize}
\item Digital pin to input or output (programmable) - 17 pins;	
\item Analog input pins or digital input / output (programmable) - 6 pins;
\item Power pins (gnd, 5V, 3.3V) - 9 pins;
\end{itemize}

The first item on the list are the useful pins. They are the ones that are available for a programmer use. Is through these pins that the Raspberry Pi is coupled to the sensors to capture environmental information. Among the 17 digital input / output pins there are 2 pins that match the serial communication module UART (Universal asynchronous receiver/transmitter). This module allows communication between a computer (for example) and the Raspberry Pi (see Special Pins section). All pins have more than one function, pins can be input or output, and the roles of each are defined in the code of possible programs that can be run on the RPi.

\paragraph{Digital Inputs}
All 17 programable pins can be used as digital inputs. When a pin is programmed to function as digital input, you use a command that, when executed, performs the "reading" of the voltage applied to it. Then, after running this command, you can know if the pin is in a "high" or "low" state (on or off).
\newline
\newline
From the electrical point of view, the program can tell if a pin is fed with 0 (zero) or 5 Volts. This function is usually used to identify whether a button is pressed or a sensor is capturing some environmental information. Note that the digital input function delivers only the values 0 or 1 (no voltage or with voltage). You can not know how much voltage is being applied to the pin.


\paragraph{Analog Inputs}
The Raspberry Pi does not have analog pins as Arduino, but is possible to use a traditional analog-to-digital converter(ADC) chip connecting to the R-Pi via SPI or I2C. For example the Microchip MCP3424 (I2C interface) which has 4 differential inputs. All ADC should work on the R-Pi, except that the timing is not as good as a microcontroller due to the available Linux versions not being true real-time. 

\paragraph{Digital outputs}
With a digital output is possible only two types of values (0 or 5 volts, 1 or 0, etc.). From a pin programmed as digital output is possible for example light up an LED, connecting a relay, trigger a motor, etc. You can program the RPi for all 17 digital outputs.

\paragraph{Special Pins}
RPi pins have some special characteristics which can be used from the software functions encoded by a software programmer. Are they:
%http://pi.gadgetoid.com/pinout
\begin{itemize}

\item PWM - Pulse Width Modulation: Treated as analog output, it is actually a digital output that generates an alternating signal (between 0 and 1) where the time that the pin is in level 1 (on) is controlled. It is used, for example, for engine speed control, or generate voltages with values controlled by the program. For the PWM communication may use pin 12;

\item UART - Universal asynchronous receiver/transmitter: One pin(TxD) can be used to transmit and another(RxD) to receive data in serial asynchronous format. For example, you can connect a data transmission module via bluetooth and allow communication with the Arduino remotely. Are reserved for UART pins 15 (RXD receives data) and 14 (TXD sends data);%http://pi.gadgetoid.com/pinout/uart

\item SPI Port - Serial Peripheral Interface: These are pins that allow synchronous serial communication faster than UART. They allow for example to connect memory cards (SD) and many other things. Are used for this purpose the pins 19 as Master Output, Slave Inpu(MOSI), 21 as Master Input, Slave Output(MISO), 23 as Serial Clock(SCLK), 24 as Chip Select0(CE0) e 26 as Chip Select1(CE1).
%http://www.raspberrypi.org/documentation/hardware/raspberrypi/spi/README.md

\item I2C Bus - Inter-Integrated Circuit: The I2C bus allows multiple devices to be connected to the Raspberry Pi, each with a unique address, that can often be set by changing jumper settings on the module. It is very useful to be able to see which devices are connected to the RPi as a way of making sure everything is working.Are used for this purpose the pins 3 for Serial Data Line(SDA) and pin 5 for Serial Clock Line(SCL)
%https://learn.adafruit.com/adafruits-raspberry-pi-lesson-4-gpio-setup/configuring-i2c
%http://en.wikipedia.org/wiki/I%C2%B2C
\end{itemize}

\subsection{Software}


\subsubsection{Operating systems}
Raspberry Pi primarily uses Linux-kernel-based operating systems.

The ARM11 chip at the heart of the Pi (pre-Pi 2) is based on version 6 of the ARM. The current releases of several popular versions of Linux, including Ubuntu,[65] will not run on the ARM11. It is not possible to run Windows on the original Raspberry Pi, though the new Raspberry Pi 2 will be able to run Windows 10.[67] The Raspberry Pi 2 currently only supports Ubuntu Snappy Core, Raspbian, OpenELEC and RISC OS.
\newline
\newline
The install manager for the Raspberry Pi is NOOBS. The operating systems included with NOOBS are:

\begin{itemize}

\item Archlinux ARM;
\item OpenELEC;
\item Pidora (Fedora Remix);
\item Puppy Linux;
\item Raspbmc and the XBMC open source digital media center;
\item RISC OS – The operating system of the first ARM-based computer;
\item Raspbian (recommended for Raspberry Pi and used in this project)[74] – Maintained independently of the Foundation; based on the ARM hard-float (armhf) Debian 7 'Wheezy' architecture port originally designed for ARMv7 and later processors (with Jazelle RCT/ThumbEE, VFPv3, and NEON SIMD extensions), compiled for the more limited ARMv6 instruction set of the Raspberry Pi. A minimum size of 4 GB SD card is required. There is a Pi Store for exchanging programs.[77]
\begin{itemize}
\item The Raspbian Server Edition is a stripped version with fewer software packages bundled as compared to the usual desktop computer oriented Raspbian.[78][79]
\item The Wayland display server protocol enable the efficient use of the GPU for hardware accelerated GUI drawing functions.[80] on 16 April 2014 a GUI shell for Weston called Maynard was released.
\item PiBang Linux is derived from Raspbian.[81]
\item Raspbian for Robots - A fork of Raspbian for robotics projects with LEGO, Grove, and Arduino.[82]
\end{itemize}
%[65]Gareth Halfacree. "Raspberry Pi review: Eben Upton reveals all". linuxuser.co.uk.http://www.linuxuser.co.uk/features/raspberry-pi-interview-eban-upton-reveals-all
%[67] Dallas, Kevin (2 February 2015). "Windows 10 Coming to Raspberry Pi 2". Building Apps for Windows. Microsoft. Retrieved 2 February 2015.http://blogs.windows.com/buildingapps/2015/02/02/windows-10-coming-to-raspberry-pi-2/
%[74]"Raspbian – Debian optimized for the Raspberry Pi hardware".http://www.raspbian.org/
%[77]"Introducing the Pi Store". Retrieved 16 September 2014.http://www.raspberrypi.org/introducing-the-pi-store/
%[78]Yau, Lawrence. "Raspbian Server Edition Version 2.4". The Rantings and Ravings of a Madman. sirlagz.net - Lawrence Yau. Retrieved 4 July 2013.http://sirlagz.net/2013/06/13/raspbian-server-edition-2-4/
%[79]Raspbian wheezy". Downloads. Raspberry Pi Foundation. Retrieved 10 January 2013.http://www.raspberrypi.org/downloads
%[80]Eben Upton (24 May 2013). "Wayland". Raspberry Pi. Retrieved 25 May 2013.http://www.raspberrypi.org/archives/4053
%[81]"Inspired by CrunchBang Linux, and based on Raspbian. Features the Openbox desktop environment".http://www.pibanglinux.org/
%[82]"A fork of Raspbian for robotics projects with LEGO, Grove, and Arduino.".http://sourceforge.net/projects/dexterindustriesraspbianflavor/
\end{itemize}

A list of other operating systems that can be installed on Raspberry Pi but are not included with NOOBS can be found in the appendix.
%"Other operating systems" http://en.wikipedia.org/wiki/Raspberry_Pi#cite_note-RaspbianServerEdition-78 

\subsubsection{WiringPi}
WiringPi is a library for access to GPIO interface written in C. Its use can be performed with C, C ++, or other programming languages through wrappers (WIRING PI, 2014). Wrapper is an outer layer that extends WiringPi and can be implemented in different programming languages. This will allow the programmer to carry out projects not only in C or C ++, but also in language implemented by the wrapper. There are wrappers being developed in various languages such as Java, Ruby and Pyton. This last will be used in this paper from the wrapper raspberry-gpio-python.

\subsubsection{Python}
Python is considered a very high level language because its syntax is simple and its dynamic typingin addition to being interpreted which makes it great for scripting and robust for various paradigms including object orientation. With all these benefits of language, Python still surprises to be an open source software being available for all major operating systems.
[] % [http://www.enucomp.com.br/2012/conteudos/artigos/arduinoPython.pdf]


\subsubsection{GPIO in Python}
The easiest way to control the GPIO pins is using the RPi.GPIO Python library. The RPi.GPIO module is installed by default in Raspbian, but if needed, installing the library is easy if followed the RPi.GPIO Installation Guide. Once installed using the pins is as easy as below: 


%\begin{lstlisting}[language=Python]
\renewcommand{\theFancyVerbLine}{
  \sffamily\textcolor[rgb]{0.5,0.5,0.5}{\scriptsize\arabic{FancyVerbLine}}}
\begin{minted}[mathescape,
               linenos,
               numbersep=0pt,
               gobble=0,
               frame=lines,
               framesep=2mm]{python}

  import RPi.GPIO as GPIO

  # Use GPIO numbers, not pin numbers
  GPIO.setmode(GPIO.BCM) 

  # set up GPIO channels - one input and one output
  GPIO.setup(7, GPIO.IN) 
  GPIO.setup(8, GPIO.OUT)

  # input from GPIO7
  input_value = GPIO.input(7)

  # output to GPIO8
  GPIO.output(8, True)
\end{minted}
%\end{lstlisting}
%\newline
%\newline
Any RPi.GPIO script must be run as root because this library needs to acces /dev/mem and for safety reasons, it is not recommended to provide access permission to this directory.

%installation guide: http://sourceforge.net/p/raspberry-gpio-python/wiki/install/
%\begin{figure}[h]
%    \includegraphics[width=0.57\textwidth,natwidth=610,natheight=642]{pictures/gpio.png}
%    \caption{controlling gpio with Python}
%\end{figure}


\paragraph{GPIO.BOARD and GPIO.BCM}
The GPIO.BOARD option specifies that you are referring to the pins by the number of the pin the the plug - i.e the numbers printed on the board (e.g. P1) and in the middle of the diagram on figure 5.
\begin{figure}[h]
    \includegraphics[width=0.45\textwidth,natwidth=610,natheight=642]{pictures/ModelAB.png}
    \caption{GPIO pins}
\end{figure}
\newline
\newline
The GPIO.BCM option means that you are referring to the pins by the "Broadcom SOC channel" number, these are the numbers after "GPIO" in the green rectangles around the outside of the diagram above.
\newline
\newline
Unfortunately the BCM numbers changed between versions of the Model B, and you'll need to work out which one you have guide here. So it may be safer to use the BOARD numbers if you are going to use more than one pi in a project.

\paragraph{Pigpio}
%http://abyz.co.uk/rpi/pigpio/python.html
Pigpio is a Python module for the Raspberry which talks to the pigpio daemon to allow control of the general purpose input outputs (GPIOs). Pigpio Python scripts may be run on Windows, Macs, and Linux machines. Only the pigpio daemon needs to be running on the Pi.
\newline
\newline
Pigpio provides all the standard gpio features and in addition it provides hardware timed PWM suitable for servos, LEDs, and motors and samples/timestamps gpios 0-31 up to 1 million times per second (default 200 thousand).

\section{JUSTIFICATION}
As with other platforms, Raspberry Pi allows coupling several sensors whose handling can be made from raspberry-gpio-python or any other API available. On other platforms, such as the Arduino, the APIs provide libraries that encapsulate the complexity of implementation and offer only the interface to use. These libraries do not yet exist formally for those who want to use Python as a development language for Raspberry Pi.
\newline
\newline
This may be a consequence of run under the Linux kernel which is not suitable for real time applications - it is multitasking O/S and another process may be given priority over the CPU, causing jitter in the program[??].

\section{RELATED WORK}
As mentioned in section ?? there is no specific library to work with python applied to Raspberry IP, which provides a wide variety of sensors but there are some projects of such libraries for Arduino, and libraries for a particular set of sensors, which will be shown below.
\newline
\newline
%%%%%%%%%%%%%%%%%%%%%%%%%%%%%%%%%%%%%%%%%%%%%%%%%%%%%%%%%%%%%%%%%%%%%%%%%%%%%%%%%%%%%%%%%%%%%%%%%%%%%%%%
PrivateEyePi[] is a project developed for security/automation that uses binds programming and eletronics.%[http://projects.privateeyepi.com/]
This is an open source project that is free of charge and can be copied, shared and modified without restriction.
The user can use the Raspberry Pi or an tiny wireless Arduino to connect sensors and send data over the Internet.
\newline
\newline
PrivateEyePi Provides tutorials for users explaining how to build, wire, convert the sensor to a wireless battery operated IOT device and how to connect it to the Internet. The project also provides a cloud based alarm system where the client can group sensors using zones. Zones can be activated and alarms triggered based on rules that can be defined.
\newline
\newline
Some sensors, like relay switches, can be controlled through The Internet. users can also control the alarm system through the PrivateEyePi web based dashboard, which allows monitor the status of sensors and view temperature and humidity readings in real time from The Internet. Historical information is provided by the analytics module. The library provides also a sophisticated rules engine that permit create rules that are processed in real time to create alerts. Those methods require parameters like sensor values, time of day, days of week, alarm activated/deactivated to define rules specific to individual sensors.     
Sensors that PrivateEyePi provides PIR motion sensor, DS18B20 digital thermometer, DHT22 for temperature and humidity readings and a generic water sensor.
\begin{figure}[h]
    \includegraphics[width=0.5\textwidth,natwidth=610,natheight=642]{pictures/dashboardPrivateeye.png}
    \caption{PrivateEyePi web based dashboard}
\end{figure}
\newline
\newline
%%%%%%%%%%%%%%%%%%%%%%%%%%%%%%%%%%%%%%%%%%%%%%%%%%%%%%%%%%%%%%%%%%%%%%%%%%%%%%%%%%%%%%%%%%%%%%%%%%%%%%%%
%[Jefferson Jardem Izaias de Souza1 , Luís Bruno Pereira do Nascimento1 , Paulo Rodrigues dos Santos Filho2]
[] describes a library that uses python called pySerial using serial port to communicate with Arduino mainly, but also with Python running on Windows, Linux, BSD (possibly any POSIX compliant system), Jython and IronPython (.NET and Mono). It encapsulates the access for the serial port and provides backends for Python. The module named "serial" automatically selects the appropriate backend.
\newline
\newline
Features:
\begin{itemize}
\item Same class based interface on all supported platforms.
\item Access to the port settings through Python properties.
\item Support for different byte sizes, stop bits, parity and flow control with RTS/CTS(Request to Send / Clear to Send) and/or Xon/Xoff.
\item Working with or without receive timeout.
\item File like API with “read” and “write” ("readline" etc. also supported).
\item The files in this package are 100\% pure Python.
\item The port is set up for binary transmission. No NULL byte stripping, CR-LF translation(carriage return-linefeed) etc. (which are many times enabled for POSIX.) This makes this module universally useful.
\item Compatible with io library (Python 2.6+)
\item RFC 2217 client (experimental), server provided in the examples
\end{itemize}

This work [??] describes the use of LM35 (temperature sensor) and a Light Dependent Resistor.
\newline
\newline
%%%%%%%%%%%%%%%%%%%%%%%%%%%%%%%%%%%%%%%%%%%%%%%%%%%%%%%%%%%%%%%%%%%%%%%%%%%%%%%%%%%%%%%%%%%%%%%%%%%%%%%%
%http://playground.arduino.cc/Main/DHTLib
Arduino DHTLib [??] is a library for reading temperature and humidity from the sensors of HDT11\'s famly, such as DHT11, DHT21, DHT22 DHT33 e DHT44, applied to Arduino. 
\newline
\newline
The DHT11/21/22 has three lines, GND, +5V and a single data line. By means of a handshake the values are clocked out over the single digital line. Handshake for DHT21/22 is identical, but data format are different. The library is rewritten from scratch and is not compatible with earlier DHT libraries to be able to support both DHT's and stay as simple as possible and to minimize footprint.
\newline
\newline
The interface supports only one function for reading the humidity and temperature from the sensors and store it in two members of the class. The read() function verifies the checksum of the data transmission and it has a time out function. If there is a checksum error the values of temperature and/or humidity might still be valid.
\newline
\newline
The class has 6 read functions read11(PIN), read(PIN) and readxx(PIN) which have essentially the same interface. They read the DHT connected to PIN, and fill the two class members temperature and humidity. Multiple reads from these class members (Humidity and Temperature) will return the same (previous) values until a new read is done.
\newline
\newline
The \textit{readXX()} functions return:

\begin{itemize}
\item DHTLIB\_OK (0) : if the sensor sample and its checksum is OK.
\item DHTLIB\_ERROR\_CHECKSUM (-1) : if the checksum test failed. This means that data was received but may be incorrect.
\item DHTLIB\_ERROR\_TIMEOUT (-2) : if a timeout occured, communication failed.
\end{itemize}

In case of a DHTLIB\_ERROR\_TIMEOUT, humidity and temperature will get the value DHTLIB\_INVALID\_VALUE. In case of DHTLIB\_ERROR\_CHECKSUM the values of humidity and temperature are left unchanged as it is impossible to determine which byte failed in the checksum. It is up to the programmer to decide what to do. One can compare with previous value, but better reread the sensor.
\newline
\newline
%%%%%%%%%%%%%%%%%%%%%%%%%%%%%%%%%%%%%%%%%%%%%%%%%%%%%%%%%%%%%%%%%%%%%%%%%%%%%%%%%%%%%%%%%%%%%%%%%%%%%%%%
%https://code.google.com/p/arduino-new-ping/
NewPing Library for Arduino[] is an ultrasonic sensor library for arduino that was developed to work with the sensors SR04, SRF05, SRF06, DYP-ME007 and Parallax PING)))\textsuperscript{TM}. It is written with c++ and intended to use with Sketches (Software written using Arduino). 
\newline
\newline
Features:
\begin{itemize}
\item Works with many different ultrasonic sensor models: SRF05, SRF06, DYP-ME007, Parallax PING)))\textsuperscript{TM} and SR04.
\item Option to interface with all but the SRF06 sensor using only one Arduino pin.
\item Doesn't lag for a full second if no ping echo is received like all other ultrasonic libraries.
\item Ping sensors consistently and reliably at up to 30 times per second.
\item Timer interrupt method for event-driven sketches.
\item Built-in digital filter method ping\_median() for easy error correction.
\item Uses port registers when accessing pins for faster execution and smaller code size.
\item Allows setting of a maximum distance where pings beyond that distance are read as no ping "clear".
\item Ease of using multiple sensors (example sketch that pings 15 sensors).
\item More accurate distance calculation (cm, inches and microseconds).
\item Doesn't use pulseIn, which is slow and gives incorrect results with some ultrasonic sensor models.
\item Actively developed with features being added and bugs/issues addressed. 
\end{itemize}

%%%%%%%%%%%%%%%%%%%%%%%%%%%%%%%%%%%%%%%%%%%%%%%%%%%%%%%%%%%%%%%%%%%%%%%%%%%%%%%%%%%%%%%%%%%%%%%%%%%%%%%%
%https://github.com/adafruit/Adafruit-Raspberry-Pi-Python-Code
Adafruit's Raspberry-Pi Python Code Library[] is the most closed work with this paper. It is a growing collection of libraries and example python scripts written by Limor Fried, Kevin Townsend and Mikey Sklar under BSD license for controlling a variety of Adafruit electronics with a Raspberry Pi.
\newline
\newline
This library provides a collection of python scripts to work with sensors, providing classes that can be used to measure values from the sensor that the class implements. 
\newline
\newline
Despite being a library with a considerable amount of sensors, if the user wants to change the type of sensor in a project, it is necessary review the documentation of the new sensor to be used because the library does not use a standard for equivalent sensors.




\section{APPLICATION DEVELOPMENT}

Many small embedded systems exist to collect data from sensors, analyse the data, and either take an appropriate action or send that sensor data to another system for processing. One of the many challenges of embedded systems design is the fact that parts that are used today may be out of production tomorrow, or system requirements may change and may will be needed to choose a different sensor down the road.
\newline
\newline
Creating new drivers is a relatively easy task, but integrating them into existing systems is both error prone and time consuming since sensors rarely use the exact same units of measurement. By reducing all data to a single sensors/event family type and settling on specific, standardised SI units for each sensor family the same sensor types return values that are comparable with any other similar sensor. This enables users to switch sensor models with very little impact on the rest of the system, which can help mitigate some of the risks and problems of sensor availability and code reuse.
\newline
\newline
%This Library provides a common interface and data type for any supported sensor. It defines some basic information about the sensor (sensor limits, etc.), and returns standard SI units of a specific type and scale for each supported sensor type.
%\newline
%\newline
LibsensorPy provides a simple abstraction layer between user's application and the actual sensor Hardware, allowing to drop in any comparable sensor with only one or two lines of code to change in the project that uses the library. This change is essentially in the constructor since the functions to read data and get information about the sensor are defined in the family sensor class, e.g. UltrasonicSensor class.
\newline
\newline
This is imporant useful for two reasons:

\begin{enumerate}
\item Users can use the data right away because it's already converted to SI units that is understandable and can compare, rather than meaningless values like 0..1023.
\item Because SI units are standardised in the sensor library, users can also do quick sanity checks working with new sensors, or drop in any comparable sensor if needed better sensitivity or if a lower cost unit becomes available, etc.
\end{enumerate}

Light sensors will always report units in lux, pressure sensors will always report units in hPa and so forth, freeing user up to focus on the data, rather than digging through the datasheet to understand what the sensor's raw numbers really mean. Also the library offer methods to convert the standard SI to other measurement unit, thus who are using the library can abstracts these conversions. A sheet of Standardised SI values and measurable units and conversions for each sensor can be seen in appendix XXXX 

\subsection{Method/Methodology}

\subsection{Functional Requirements}
 
\subsection{Non-functional Requirements}

\subsection{Why not use Java}

\subsection{Architecture}

\subsection{tests}

\section{CONCLUSION AND FUTURE WORK}
This work presents an extensible open source library, available at Python Package Index (PyPI), the official third-party repository for the Python programming language [] %https://pypi.python.org/packages/source/l/libsensorPy/libsensorPy-0.0.1.tar.gz
and on Github [https://github.com/juniorug/libsensorPy], the Git web-based repository hosting service, Which offers all of the distributed revision control and source code management (SCM) functionality of Git as well as adding its own features.
\newline
\newline
Its goal is to facilitate the creation of ubiquitous systems.The tool is designed so that facilitate the inclusion of new sensors, families and factories of sensors, as well as ease of use by the user, abstracting technical and behaviors specific to that type of system, using design patterns and following the SOLID principles.
\newline
\newline
Some suggestions can be seen below:
\begin{itemize}
\item Test the sensors that have been implemented but have not been tested;
\item Add new sensors to the library;

\end{itemize}
\section{REFERENCES}

\subsection{Citations}

\subsection{Tables}

\subsection{Figures}

\subsection{Theorem-like Constructs}


%
% The following two commands are all you need in the
% initial runs of your .tex file to
% produce the bibliography for the citations in your paper.
\bibliographystyle{abbrv}
\bibliography{sigproc}  % sigproc.bib is the name of the Bibliography in this case
% You must have a proper ".bib" file
%  and remember to run:
% latex bibtex latex latex
% to resolve all references
%
% ACM needs 'a single self-contained file'!
%
%APPENDICES are optional
%\balancecolumns
\appendix
%Appendix A
\section{Headings in Appendices}
The rules about hierarchical headings discussed above for
the body of the article are different in the appendices.
In the \textbf{appendix} environment, the command
\textbf{section} is used to
indicate the start of each Appendix, with alphabetic order
designation (i.e. the first is A, the second B, etc.) and
a title (if you include one).  So, if you need
hierarchical structure
\textit{within} an Appendix, start with \textbf{subsection} as the
highest level. Here is an outline of the body of this
document in Appendix-appropriate form:
\subsection{Introduction}
\subsection{The Body of the Paper}
\subsubsection{Type Changes and  Special Characters}
\subsubsection{Math Equations}
\paragraph{Inline (In-text) Equations}
\paragraph{Display Equations}
\subsubsection{Citations}
\subsubsection{Tables}
\subsubsection{Figures}
\subsubsection{Theorem-like Constructs}
\subsubsection*{A Caveat for the \TeX\ Expert}
\subsection{Conclusions}
\subsection{Acknowledgments}
\subsection{Additional Authors}
This section is inserted by \LaTeX; you do not insert it.
You just add the names and information in the
\texttt{{\char'134}additionalauthors} command at the start
of the document.
\subsection{References}
Generated by bibtex from your ~.bib file.  Run latex,
then bibtex, then latex twice (to resolve references)
to create the ~.bbl file.  Insert that ~.bbl file into
the .tex source file and comment out
the command \texttt{{\char'134}thebibliography}.
% This next section command marks the start of
% Appendix B, and does not continue the present hierarchy
\section{More Help for the Hardy}
The acm\_proc\_article-sp document class file itself is chock-full of succinct
and helpful comments.  If you consider yourself a moderately
experienced to expert user of \LaTeX, you may find reading
it useful but please remember not to change it.
\balancecolumns
% That's all folks!
\end{document}
